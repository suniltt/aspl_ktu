
\documentclass[a4paper,11pt]{book}
\usepackage[T1]{fontenc}
\usepackage[utf8]{inputenc}
\usepackage{lmodern}
\usepackage{hyperref}
\usepackage{graphicx}
\usepackage[english]{babel}
 
\usepackage{listings}
\usepackage{color}

\definecolor{dkgreen}{rgb}{0,0.6,0}
\definecolor{gray}{rgb}{0.5,0.5,0.5}
\definecolor{mauve}{rgb}{0.58,0,0.82}

\lstset{frame=tb,
  language=Java,
  aboveskip=3mm,
  belowskip=3mm,
  showstringspaces=false,
  columns=flexible,
  basicstyle={\small\ttfamily},
  numbers=none,
  numberstyle=\tiny\color{gray},
  keywordstyle=\color{blue},
  commentstyle=\color{dkgreen},
  stringstyle=\color{mauve},
  breaklines=true,
  breakatwhitespace=true,
  tabsize=3
}
\newenvironment{dedication}
{
   \cleardoublepage
   \thispagestyle{empty}
   \vspace*{\stretch{1}}
   \hfill\begin{minipage}[t]{0.66\textwidth}
   \raggedright
}
{
   \end{minipage}
   \vspace*{\stretch{3}}
   \clearpage
}

%%%%%%%%%%%%%%%%%%%%%%%%%%%%%%%%%%%%%%%%%%%%%%%%
% Chapter quote at the start of chapter        %
% Source: http://tex.stackexchange.com/a/53380 %
%%%%%%%%%%%%%%%%%%%%%%%%%%%%%%%%%%%%%%%%%%%%%%%%
\makeatletter
\renewcommand{\@chapapp}{}% Not necessary...
\newenvironment{chapquote}[2][2em]
  {\setlength{\@tempdima}{#1}%
   \def\chapquote@author{#2}%
   \parshape 1 \@tempdima \dimexpr\textwidth-2\@tempdima\relax%
   \itshape}
  {\par\normalfont\hfill--\ \chapquote@author\hspace*{\@tempdima}\par\bigskip}
\makeatother

%%%%%%%%%%%%%%%%%%%%%%%%%%%%%%%%%%%%%%%%%%%%%%%%%%%
% First page of book which contains 'stuff' like: %
%  - Book title, subtitle                         %
%  - Book author name                             %
%%%%%%%%%%%%%%%%%%%%%%%%%%%%%%%%%%%%%%%%%%%%%%%%%%%

% Book's title and subtitle
\title{\Huge \textbf{Applied Probability and Statistics Lab}   \\ \huge For KTU MCA }
% Author
\author{\textsc{Dr.Sunil Thomas Thonikuzhiyil}\thanks{\url{College of Engineering Attingal}} \\  }


\begin{document}

\frontmatter
\maketitle

%%%%%%%%%%%%%%%%%%%%%%%%%%%%%%%%%%%%%%%%%%%%%%%%%%%%%%%%%%%%%%%
% Add a dedication paragraph to dedicate your book to someone %
%%%%%%%%%%%%%%%%%%%%%%%%%%%%%%%%%%%%%%%%%%%%%%%%%%%%%%%%%%%%%%%
\begin{dedication}
Dedicated to Calvin and Hobbes.
\end{dedication}

%%%%%%%%%%%%%%%%%%%%%%%%%%%%%%%%%%%%%%%%%%%%%%%%%%%%%%%%%%%%%%%%%%%%%%%%
% Auto-generated table of contents, list of figures and list of tables %
%%%%%%%%%%%%%%%%%%%%%%%%%%%%%%%%%%%%%%%%%%%%%%%%%%%%%%%%%%%%%%%%%%%%%%%%
\tableofcontents
\listoffigures
\listoftables

\mainmatter

%%%%%%%%%%%
% Preface %
%%%%%%%%%%%
\chapter*{Preface}


%\section*{Un-numbered sample section}


%\section*{Another sample section}

%\section*{Structure of book}
 

%\section*{About the companion website}
 

%%%%%%%%%%%%%%%%%%%%%%%%%%%%%%%%%%%%
% Give credit where credit is due. %
% Say thanks!                      %
%%%%%%%%%%%%%%%%%%%%%%%%%%%%%%%%%%%%
\section*{Acknowledgements}
\begin{itemize}
\item A special word of thanks goes to Professor Don Knuth\footnote{\url{http://www-cs-faculty.stanford.edu/~uno/}} (for \TeX{}) and Leslie Lamport\footnote{\url{http://www.lamport.org/}} (for \LaTeX{}).
 
\end{itemize}
\mbox{}\\
%\mbox{}\\
 

%%%%%%%%%%%%%%%%
% NEW CHAPTER! %
%%%%%%%%%%%%%%%%
\chapter{Visualizing Data}
 

\section*{What we will learn}
In this experiment we will try to  learn the preliminaries of R along with methods of visualizing data in R

Tables, charts and plots. Visualizing Measures of Central Tendency, Variation,
and Shape. Box plots, Pareto diagrams. How to find the mean median standard
deviation and quantiles of a set of observations.




\section{Introduction to R}
\subsection{Installing R }
\subsection{Installing R studio}
 
 \section{R fundamentals}
 
 

This is a review of R fundamentals. No details are covered. It is assumed that the participant is familiar with programming.
\subsection{Basic Math operations in R}
Type the following  on > prompt and see the results. 
 \begin{lstlisting}[language=R]
1 + 1
5 - 3
3 * 3
4 / 5
4 ^ 2   # Exponetiation
5 %% 2   # Modulus
5 %/% 2  # integer division


\end{lstlisting}

\subsection{Variables and assignment in R}
 There is no  need to declare variables in advance. Rules for variable names are almost similar to those in C programming language. There are two assignment : + and <-.  The preferred assignment operator is <- . 
 Try the following.
  \begin{lstlisting}[language=R]
x=20  # assign x=20
y<-3  # assign y =3
x     # Display x
y     # Display y
x=y
x
y<-x*10

\end{lstlisting}

A variable can be removed using rm(0 function.

 \begin{lstlisting}[language=R]
 rm()  # remove x 
 x     # try printing x after removal
\end{lstlisting}
 \section{R data structures}
 
 \section{Plotting in R}

\chapter{Probability Distributions}

\chapter{Random Samples}

\chapter{Binomial Distribution and Central Limit theorem}

\chapter{Confidence Intervals}
\chapter{Correlation}
\chapter{Regression}
   \begin{lstlisting}[language=R]
 

\end{lstlisting}
 

\end{document}
